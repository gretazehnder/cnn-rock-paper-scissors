\documentclass[12pt,a4paper]{article}

% packages
\usepackage[utf8]{inputenc}
\usepackage[T1]{fontenc}
\usepackage{lmodern}
\usepackage{graphicx}
\usepackage{amsmath}
\usepackage{amssymb}
\usepackage{geometry}
\usepackage{float}
\usepackage{hyperref} % keep hyperref last

\geometry{margin=2.5cm}

% title
\title{Rock, Paper, Scissors: CNN for Image Classification}
\author{Greta Zehnder}
\date{\today}

\begin{document}

\maketitle

% abstract
\begin{abstract}
    This project aims to investigate the application of Convolutional Neural Networks (CNNs) to the task of image classification using a Rock-Paper-Scissors (RPS) dataset, with the objective of designing, training, and evaluating multiple deep learning models. 

The experimental pipeline is composed of dataset exploration, data preprocessing (which includes train/validation/test splitting, input normalization, and data augmentation), followed by the development of three CNN models (ordered by increasing complexity), and their supervised training and performance evaluation. Finally, a generalization part is carried out to highlight the effectiveness of using CNNs for image classification tasks.

The entire study was carried out in accordance with the official TensorFlow/Keras API documentation.

\end{abstract}

% introduction
\section{Introduction}


%-----------------------------------------------------------------
\section{Data exploration and preprocessing}

\subsection{Exploratory Data Analysis}
% dataset description + EDA

\subsection{Preprocessing}

\subsubsection{Train, validation and test splitting}
% split strategy

\subsubsection{Normalization}
% rescaling

\subsubsection{Data augmentation}
% augmentation techniques

%---------------------------------------------------------------------
\section{CNN architecture and training}

\subsection{Model A: baseline CNN}
% simple architecture, no tuning

\subsection{Model B: intermediate CNN}
% increased complexity

\subsection{Model C: complex CNN}
% most complex model


%--------------------------------------------------------------------
\section{Evaluation and analysis}

\subsection{Experimental Results}
% quantitative results

\subsection{Model Comparison}
% table + metrics

\subsection{Discussion}
% interpretation, overfitting, limits


%----------------------------------------------------------
%optional generalization test
\section{Optional generalization test}

%conclusions
\section{Conclusions}
This project demonstrates...

%bibliography
\bibliographystyle{plain}
\bibliography{references}

%declaration
\section*{Declaration}
\textit{
I declare that this material, which I now submit for assessment, is entirely my own work and has not been taken from the work of others, save and to the extent that such work has been cited and acknowledged within the text of my work. I understand that plagiarism, collusion, and copying are grave and serious offences in the university and accept the penalties that would be imposed should I engage in plagiarism, collusion or copying. This assignment, or any part of it, has not been previously submitted by me or any other person for assessment on this or any other course of study.
}

\end{document}
